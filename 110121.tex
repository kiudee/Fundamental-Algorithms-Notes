%& -shell-escape
% -----------------------------------------------
%     Einstellungen zum Layout:
% -----------------------------------------------
\documentclass[12pt,oneside,a4paper,parskip=on,fleqn]{scrartcl}
\usepackage{ngerman}
\usepackage[utf8]{inputenc}
\usepackage{amsmath}
\usepackage{amssymb}
\usepackage{amsthm}
\usepackage{stmaryrd}
\usepackage{enumerate}
\usepackage{jeffe}
\usepackage{pgf}
\usepackage{tikz}
\usepackage{todonotes}

\usetikzlibrary{arrows,automata,positioning}
\setlength\parindent{0pt}   % Festlegen des Absatzeinzuges

\newcounter{excnt}
\newtheorem{exer}[excnt]{Exercise}

\begin{document}
\section*{Vorlesung 21.01.2011 - StringMatching}
\begin{proof}[Lemma 28] Running Time of ComputePrefixFunction is $\mathcal{O}(m)$\\
	Sketch: Potential $\Phi(\mathrm{currentstate}):=k$
	\begin{figure}[ht]
	\centering
	\begin{tikzpicture}
		\draw (0,0) -- (5,0);
		\draw (0,0) -- (0,5);

		\draw (0.5,0) -- node[left]{$1$} (0.5,0.5);
		\draw (0.5,0.5) -- (1,0.5);
		\draw (1,0.5) -- node[left]{$1$}(1,1);
		\draw (1,1) -- (1.5,1);
		\draw (1.5,1) -- node[left]{$1$}(1.5,1.5);
	\end{tikzpicture}
	\end{figure}
	\todo[inline]{Grafik vervollständigen}
	Always: $k\geq 0$ \\
	Initially: $k=0$\\
	Total increase of $k$: $\leq m-2$\\
	$\Ra$ total number of decrease-executions of „$k=\pi(k)$“ $\leq m-1$\\
	$\Ra \mathcal{O}(m)$
\end{proof}

\begin{proof}[Lemma 29 Sketch] 
	„$<$“ $i\in \pi^*(q) \Ra i\in \pi^u(q),\ u\in \N_0$\\
	\begin{description}
		\item[(IB)] $u=0 \Ra i=q \checkmark$
		\item[(IS)] $P_{\pi(i)} \sqsupset P_i \underset{\text{IH}}{\sqsupset} P_q$
	\end{description}

	„$>$“ Suppose $\exists j \in \bigl\{ k | P_k \sqsupset P_q \bigr\}\setminus \pi^*(q).$ Wlog. $j$ maximal.
	\[
		q \in \bigl\{ k| P_k \sqsupset P_q \bigr\} \cap \pi^*(q) \Ra j<q
	\]\[
		j' = \min\bigl\{ r\in \pi^*(q) | r>j \bigr\}
	\]
	Then \[
	\left.\begin{cases}
		P_j \sqsupset P_q \text{ since } j\in \bigl\{ k|P_k \sqsupset P_q \bigr\}\\
		P_j' \sqsupset P_q \text{ since } j'\in \pi^*(q) \text{ and } \mathrm{„<}\text{“}
	\end{cases}\right\} \underset{\text{L25a}}{\Ra} P_j \sqsupset P_{j'}
	\]
	$j$ max, $\pi(j') = j \Ra j\in\pi^*(q) \lightning$
\end{proof}

\begin{proof}[Corollary 9]
	If $r=\pi(q),$ then $P_r \sqsupset P_q$ and thus $r\geq 1$ implies $p_r=p_q$.\\
	By Lemma 30, if $r\geq 1$ then \begin{align*}r&=1+\max\bigl\{ k\in \pi^*(q-1) | p_{k+1} = p_q \bigr\}\\
	&= 1+\max\bigl\{ k|k\in E_{q-1} \bigr\} \text{ and } E_{q-1} \neq \emptyset
	\end{align*}
	If $r=0$, there is no $k\in \pi^*(q-1)$ for which we can extend $P_k$ to $P_{k+1}$ and get a suffix of $P_q$. Since then $\pi(q) > 0$. Thus $E_{q-1} = \emptyset$.
\end{proof}

\begin{proof}[Corollary 10]
	$\pi(1) = 0\checkmark$

	At the start of each iteration of the for-loop we have $k=\pi(q-1)$. This is maintained as an invariant. The while-loop searches through all values $k\in \pi^*(q-1)$ until one is found for which $p_{k+1} = p_q$. At that point $k=\max \{E_{q-1}\}$, so by Corollary 9 we can set $\pi(q)$ to $k+1$. If no such $k$ is found, $\pi(q)$ is correctly set to $0$.
\end{proof}
\end{document}