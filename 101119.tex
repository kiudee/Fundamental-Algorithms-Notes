% -----------------------------------------------
%     Einstellungen zum Layout:
% -----------------------------------------------
\documentclass[12pt,oneside,a4paper,parskip=on,fleqn]{scrartcl}
\usepackage{ngerman}
\usepackage[utf8]{inputenc}
\usepackage{amsmath}
\usepackage{amssymb}
\usepackage{amsthm}
\usepackage{enumerate}
\setlength\parindent{0pt}   % Festlegen des Absatzeinzuges
\renewcommand\qedsymbol{$\blacksquare$} % QED-Symbol in Schwarz

\begin{document}
\begin{proof}[Proof of Lemma 10]
	$d$ valid: $\quad d(u) \leq d(v) + 1\quad \forall (u,v)\in E_f.$\\
	(by induction on the number of augment + retreat operations.)

	(IB) $d(u) = \delta_G(u,t) \underset{f\equiv 0}{=} \delta_{G_f}(u,t)$ implies $d(u)\leq d(v) + 1 \quad \forall (u,v)\in E_f.$

	(IS)
	\begin{enumerate}[a)]\item augmentation along $(u,v) \in E_f.$
		\begin{itemize}
			\item remove $(u,v)$ from $E_f$
			\item create an additional edge $(v,u)\in E_f$ and thus an additional inequality $d(v) \leq d(u) + 1$\\
			We know: $d(u) = d(v) + 1$ because $(u,v)$ is admissible.
			Thus $d(v) = d(u) - 1 < d(u) + 1$.
		\end{itemize}
		\item retreat from $u$ to $\pi(u)$ modifies $d(u)$. Let $d'(u) = \min\{ d(v) + 1|(u,v)\in E_f \}$\\
		Thus, after retreat from $u \quad d'(u) \leq d(v) + 1\quad \forall (u,v)\in E_f$.\\
		The alg retreats from $u$, when $u$ has no admissible edges $(u,v)\in E_f$.

		By (IH) we have $d(u) < d(v) + 1\quad \forall (u,v)\in E_f$ and $d'(u) > d(u)$.\\
		From this $d(w) \leq d'(u) + 1 \quad \forall (w,u)\in E_f.$
	\end{enumerate}
\end{proof}

\begin{proof}[Proof of Theorem 13]
	The algorithm terminates with  $d(s) \geq |V|$. By Note 10b there is no path from $s$ to $t$ in $G_f$. By Theorem 12 $f$ is a max flow.
\end{proof}

\begin{proof}[Proof of NM]
	The algorithm performs a retreat at node $u$ only after it passed through the complete list $N(u)$. The relabelling passes through $N(u)$ a second time, resulting in a total time of $\mathcal{O}(|N(u)|)$ for each retreat.
\end{proof}

\begin{proof}[Proof of Lemma 11]
	We show:\\
	(*) Between two consecutive saturations of an edge $(u,v)$ both $d(u)$ and $d(v)$ must increase by at least 2 units.

	\[
		\overset{d(u)}{u} \longrightarrow \overset{d(v)+1}{v} \rightsquigarrow u \leftarrow v
	\]
	now $v$ must push flow to $u$:
	\[
		\overset{d(u)+1}{u} \longleftarrow \overset{d'(v)}{v}
	\]
	i.e. $d(v)$ increased by 2
	\[
		\overset{d'(u) = d(u)+2}{u} \longleftarrow \overset{d'(v) = d(u)+1}{v}\qedhere
	\]
\end{proof}

If the algorithm retreats from any node $u$ at most $k$ times, we can conclude that any edge can be saturated at most $\not\frac{k}{2} \rightsquigarrow k$ times. $\rightsquigarrow$ at most $k|E|$ saturations.

\begin{proof}[Sketch of Proof of Lemma 12]
	Each update (retreat from $u$) of $d(u)$ increases $d(u)$ by at least 1. Nodes $u$ with $d(u) \geq |V|$ are never considered again. $\rightsquigarrow$ a)

	b) L11 and L12a) $\Rightarrow$ algorithm saturates at most $\frac{|V|\cdot |E|}{\not2}$ edges. Since each augmentation saturates at least 1 edge:\\
	\# augmentations $\leq |V|\cdot |E|$
\end{proof}

\begin{proof}[Proof of Theorem 14]
		$n = |V|, m=|E|\quad$  L12, NM: the total effort to find admissable edges and to update the distance labels is $\mathcal{O}(nm)$. By L12:\\
		Number of augmentations $\leq nm$, each of which takes $\mathcal{O}(n)$ time. So the total effort for augmentation is $\mathcal{O}(n^2m)$. The total number of retreats is $\mathcal{O}(n^2)$. Each advance adds an edge to $P$, each retreat deletes an edge from $P$. Since $P$ has at most $n$ edges, the alg. requires at most $\mathcal{O}(n^2 + n^2\cdot m)$ advance operations.
\end{proof}
\end{document}
